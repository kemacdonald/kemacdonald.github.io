%%%%%%%%%%%%%%%%%%%%%%%%%%%%%%%%%%%%%%%%%%%%%%%%%%%%%%%%%%%%%%%%%%%%%%%%
%%%%%%%%%%%%%%%%%%%%%% Simple LaTeX CV Template %%%%%%%%%%%%%%%%%%%%%%%%
%%%%%%%%%%%%%%%%%%%%%%%%%%%%%%%%%%%%%%%%%%%%%%%%%%%%%%%%%%%%%%%%%%%%%%%%

%%%%%%%%%%%%%%%%%%%%%%%%%%%%%%%%%%%%%%%%%%%%%%%%%%%%%%%%%%%%%%%%%%%%%%%%
%% NOTE: If you find that it says                                     %%
%%                                                                    %%
%%                           1 of ??                                  %%
%%                                                                    %%
%% at the bottom of your first page, this means that the AUX file     %%
%% was not available when you ran LaTeX on this source. Simply RERUN  %%
%% LaTeX to get the ``??'' replaced with the number of the last page  %%
%% of the document. The AUX file will be generated on the first run   %%
%% of LaTeX and used on the second run to fill in all of the          %%
%% references.                                                        %%
%%%%%%%%%%%%%%%%%%%%%%%%%%%%%%%%%%%%%%%%%%%%%%%%%%%%%%%%%%%%%%%%%%%%%%%%

%%%%%%%%%%%%%%%%%%%%%%%%%%%% Document Setup %%%%%%%%%%%%%%%%%%%%%%%%%%%%

% Don't like 10pt? Try 11pt or 12pt
\documentclass[10pt]{article}

% The automated optical recognition software used to digitize resume
% information works best with fonts that do not have serifs. This
% command uses a sans serif font throughout. Uncomment both lines (or at
% least the second) to restore a Roman font (i.e., a font with serifs).
%\usepackage{times}
%\renewcommand{\familydefault}{\sfdefault}

% This is a helpful package that puts math inside length specifications
\usepackage{calc}
\usepackage{comment}

% Simpler bibsection for CV sections
% (thanks to natbib for inspiration)
\makeatletter
\newlength{\bibhang}
\setlength{\bibhang}{1em} %1em}
\newlength{\bibsep}
 {\@listi \global\bibsep\itemsep \global\advance\bibsep by\parsep}
\newenvironment{bibsection}%
        {\begin{enumerate}{}{%
%        {\begin{list}{}{%
       \setlength{\leftmargin}{\bibhang}%
       \setlength{\itemindent}{-\leftmargin}%
       \setlength{\itemsep}{\bibsep}%
       \setlength{\parsep}{\z@}%
        \setlength{\partopsep}{0pt}%
        \setlength{\topsep}{0pt}}}
        {\end{enumerate}\vspace{-.6\baselineskip}}
%        {\end{list}\vspace{-.6\baselineskip}}
\makeatother

% Layout: Puts the section titles on left side of page
\reversemarginpar

%
%         PAPER SIZE, PAGE NUMBER, AND DOCUMENT LAYOUT NOTES:
%
% The next \usepackage line changes the layout for CV style section
% headings as marginal notes. It also sets up the paper size as either
% letter or A4. By default, letter was used. If A4 paper is desired,
% comment out the letterpaper lines and uncomment the a4paper lines.
%
% As you can see, the margin widths and section title widths can be
% easily adjusted.
%
% ALSO: Notice that the includefoot option can be commented OUT in order
% to put the PAGE NUMBER *IN* the bottom margin. This will make the
% effective text area larger.
%
% IF YOU WISH TO REMOVE THE ``of LASTPAGE'' next to each page number,
% see the note about the +LP and -LP lines below. Comment out the +LP
% and uncomment the -LP.
%
% IF YOU WISH TO REMOVE PAGE NUMBERS, be sure that the includefoot line
% is uncommented and ALSO uncomment the \pagestyle{empty} a few lines
% below.
%

%% Use these lines for letter-sized paper
\usepackage[paper=letterpaper,
            %includefoot, % Uncomment to put page number above margin
            marginparwidth=1.2in,     % Length of section titles
            marginparsep=.05in,       % Space between titles and text
            margin=1in,               % 1 inch margins
            includemp]{geometry}

%% Use these lines for A4-sized paper
%\usepackage[paper=a4paper,
%            %includefoot, % Uncomment to put page number above margin
%            marginparwidth=30.5mm,    % Length of section titles
%            marginparsep=1.5mm,       % Space between titles and text
%            margin=25mm,              % 25mm margins
%            includemp]{geometry}

%% More layout: Get rid of indenting throughout entire document
\setlength{\parindent}{0in}

\usepackage[shortlabels]{enumitem}

%% Reference the last page in the page number
%
% NOTE: comment the +LP line and uncomment the -LP line to have page
%       numbers without the ``of ##'' last page reference)
%
% NOTE: uncomment the \pagestyle{empty} line to get rid of all page
%       numbers (make sure includefoot is commented out above)
%
\usepackage{fancyhdr,lastpage}
\pagestyle{fancy}
%\pagestyle{empty}      % Uncomment this to get rid of page numbers
\fancyhf{}\renewcommand{\headrulewidth}{0pt}
\fancyfootoffset{\marginparsep+\marginparwidth}
\newlength{\footpageshift}
\setlength{\footpageshift}
          {0.5\textwidth+0.5\marginparsep+0.5\marginparwidth-2in}
\lfoot{\hspace{\footpageshift}%
       \parbox{4in}{\, \hfill %
                    \arabic{page} of \protect\pageref*{LastPage} % +LP
%                    \arabic{page}                               % -LP
                    \hfill \,}}

% Finally, give us PDF bookmarks
\usepackage{color,hyperref}
\definecolor{darkblue}{rgb}{0.0,0.0,0.3}
\hypersetup{colorlinks,breaklinks,
            linkcolor=darkblue,urlcolor=darkblue,
            anchorcolor=darkblue,citecolor=darkblue}

%%%%%%%%%%%%%%%%%%%%%%%% End Document Setup %%%%%%%%%%%%%%%%%%%%%%%%%%%%


%%%%%%%%%%%%%%%%%%%%%%%%%%% Helper Commands %%%%%%%%%%%%%%%%%%%%%%%%%%%%

% The title (name) with a horizontal rule under it
% (optional argument typesets an object right-justified across from name
%  as well)
%
% Usage: \makeheading{name}
%        OR
%        \makeheading[right_object]{name}
%
% Place at top of document. It should be the first thing.
% If ``right_object'' is provided in the square-braced optional
% argument, it will be right justified on the same line as ``name'' at
% the top of the CV. For example:
%
%       \makeheading[\emph{Curriculum vitae}]{Your Name}
%
% will put an emphasized ``Curriculum vitae'' at the top of the document
% as a title. Likewise, a picture could be included:
%
%   \makeheading[\includegraphics[height=1.5in]{my_picutre}]{Your Name}
%
% the picture will be flush right across from the name.
\newcommand{\makeheading}[2][]%
        {\hspace*{-\marginparsep minus \marginparwidth}%
         \begin{minipage}[t]{\textwidth+\marginparwidth+\marginparsep}%
             {\large \bfseries #2 \hfill #1}\\[-0.15\baselineskip]%
                 \rule{\columnwidth}{1pt}%
         \end{minipage}}

% The section headings
%
% Usage: \section{section name}
\renewcommand{\section}[1]{\pagebreak[3]%
    \hyphenpenalty=10000%
    \vspace{1.3\baselineskip}%
    \phantomsection\addcontentsline{toc}{section}{#1}%
    \noindent\llap{\scshape\smash{\parbox[t]{\marginparwidth}{\raggedright #1}}}%
    \vspace{-\baselineskip}\par}

% An itemize-style list with lots of space between items
\newenvironment{outerlist}[1][\enskip\textbullet]%
        {\begin{itemize}[#1,leftmargin=*]}{\end{itemize}%
         \vspace{-.6\baselineskip}}

% An environment IDENTICAL to outerlist that has better pre-list spacing
% when used as the first thing in a \section
\newenvironment{lonelist}[1][\enskip\textbullet]%
        {\begin{list}{#1}{%
        \setlength{\partopsep}{0pt}%
        \setlength{\topsep}{0pt}}}
        {\end{list}\vspace{-.6\baselineskip}}

% An itemize-style list with little space between items
\newenvironment{innerlist}[1][\enskip\textbullet]%
        {\begin{itemize}[#1,leftmargin=*,parsep=0pt,itemsep=0pt,topsep=0pt,partopsep=0pt]}
        {\end{itemize}}

% An environment IDENTICAL to innerlist that has better pre-list spacing
% when used as the first thing in a \section
\newenvironment{loneinnerlist}[1][\enskip\textbullet]%
        {\begin{itemize}[#1,leftmargin=*,parsep=0pt,itemsep=0pt,topsep=0pt,partopsep=0pt]}
        {\end{itemize}\vspace{-.6\baselineskip}}

% To add some paragraph space between lines.
% This also tells LaTeX to preferably break a page on one of these gaps
% if there is a needed pagebreak nearby.
\newcommand{\blankline}{\quad\pagebreak[3]}
\newcommand{\halfblankline}{\quad\vspace{-0.5\baselineskip}\pagebreak[3]}

% Uses hyperref to link DOI
\newcommand\doilink[1]{\href{http://dx.doi.org/#1}{#1}}
\newcommand\doi[1]{doi:\doilink{#1}}

% For \url{SOME_URL}, links SOME_URL to the url SOME_URL
\providecommand*\url[1]{\href{#1}{#1}}
% Same as above, but pretty-prints SOME_URL in teletype fixed-width font
\renewcommand*\url[1]{\href{#1}{\texttt{#1}}}

% For \email{ADDRESS}, links ADDRESS to the url mailto:ADDRESS
\providecommand*\email[1]{\href{mailto:#1}{#1}}
% Same as above, but pretty-prints ADDRESS in teletype fixed-width font
%\renewcommand*\email[1]{\href{mailto:#1}{\texttt{#1}}}

%\providecommand\BibTeX{{\rm B\kern-.05em{\sc i\kern-.025em b}\kern-.08em
%    T\kern-.1667em\lower.7ex\hbox{E}\kern-.125emX}}
%\providecommand\BibTeX{{\rm B\kern-.05em{\sc i\kern-.025em b}\kern-.08em
%    \TeX}}
\providecommand\BibTeX{{B\kern-.05em{\sc i\kern-.025em b}\kern-.08em
    \TeX}}
\providecommand\Matlab{\textsc{Matlab}}

%%%%%%%%%%%%%%%%%%%%%%%% End Helper Commands %%%%%%%%%%%%%%%%%%%%%%%%%%%

%%%%%%%%%%%%%%%%%%%%%%%%% Begin CV Document %%%%%%%%%%%%%%%%%%%%%%%%%%%%

\begin{document}
\makeheading{Kyle MacDonald}

\section{Contact Information}

% NOTE: Mind where the & separators and \\ breaks are in the following
%       table.
%
% ALSO: \rcollength is the width of the right column of the table
%       (adjust it to your liking; default is 1.85in).
%
\newlength{\rcollength}\setlength{\rcollength}{1.4in}%
%
\begin{tabular}[t]{@{}p{\textwidth-\rcollength}p{\rcollength}}
%\href{http://www.cse.osu.edu/}%
{Department of Psychology} & \\
\href{https://psychology.stanford.edu/}{Stanford University} & 860-324-7315 \\
450 Serra mall  & \email{kylem4@stanford.edu} \\
Stanford, CA  94305  & \href{http://kemacdonald.com}{kemacdonald.com} \\
\end{tabular}


%%%%%%%%%%%%% Employment %%%%%%%%%%%%%%%

\vspace{.1in}

\section{Employment}

\textbf{Lab Manager} \hfill {2010-2013}
\begin{innerlist}
\item[] Language Learning Lab,\\
	Department of Psychology, Stanford University\\
        Anne Fernald, Ph.D
\end{innerlist}

\vspace{.1in}

\textbf{Research Assistant} \hfill {2008-2010}
\begin{innerlist}
\item[] Cognitive Development Lab,\\
        Department of Psychology, Wesleyan University\\
        Hilary Barth, Ph.D and Anna Shusterman, Ph.D
\end{innerlist}


%%%%%%%%%%%%% EDUCATION %%%%%%%%%%%%%%%

\section{Education}

\href{http://www.stanford.edu/}{\textbf{Stanford University}},
Stanford, CA
\begin{outerlist}

\item[] Ph.D.,
        \href{https://psychology.stanford.edu/gradprogram}
             {Developmental Psychology}, 2013 - present
        \begin{innerlist}
        \item Thesis Topic: \emph{Social information changes attention and memory during language processing and word learning}
        \item Advisor:
              \href{http://www.stanford.edu/~mcfrank/}
                   {Michael C. Frank}
        \end{innerlist}
\end{outerlist}

\vspace{.1in}

\href{http://www.wesleyan.edu/}{\textbf{Wesleyan University}},
Middletown, CT
\begin{outerlist}
\item[] B.A.,
             {with high honors in Psychology}, May 2010
        \begin{innerlist}
         \item Thesis Topic: \emph{Group membership disrupts preschoolers' selective learning}
         \item Advisor:
              \href{http://hbarth.faculty.wesleyan.edu/}
                   {Hilary Barth}
        \end{innerlist}

\end{outerlist}

%%%%%%%%%%%%% AWARDS %%%%%%%%%%%%%%%

\section{Honors and Awards}

\vspace{-.1in}

\begin{outerlist}
\item[] Stinehart-Reed Graduate Fellowship, Stanford University (2017 - 2018)
\item[] Graduate Research Fellowship, National Science Foundation (2014 - 2017)
\item[] Norman H. Anderson Travel Award, Stanford University  (2016)
\item[] High Honors in Psychology, Wesleyan University (2010)
\item[] Walkley Prize for Excellence in Psychological Research, Wesleyan University (2010)
\item[] Quantiative Analysis Center Research Fellowship, Wesleyan University (2008)
\end{outerlist}

%%%%%%%%%%%%% PUBLICATIONS %%%%%%%%%%%%%%%

\vspace{.1in}

\section{Publications}
\vspace{-.1275in}
\begin{bibsection}

\item {\bf MacDonald, K.}, LaMarr, T., Corina, D., Marchman, V.A., \& Fernald, A. (under review). Real-time lexical comprehension in young children learning American Sign Language. \emph{Developmental Science}.

\item {\bf MacDonald, K.}, Marchman, V.A., \& Fernald, A. (in prep). M-o-o-s as cues: Young children map novel animal vocalizations to unfamiliar animals. Manuscript in preparation.

\item Sanchez, A., Meylan, S., Braginsky, M., {\bf MacDonald, K.}, Yurovsky, D., \& Frank, M. C. (in prep). childes-db: a flexible and reproducible interface to the Child Language Data Exchange System (CHILDES).

\item Hardwicke, T. E., Mathur, M. B., {\bf MacDonald, K.}, Nilsonne, G., Banks, G. C., Kidwell, M.C., Hofelich-Mohr, A., Clayton, E., Yoon, E.J., Tessler, M.H., Lenne, R., Altman, S., Long, B., \& Frank, M.C. (in prep). Data availability, reusability, and analytic reproducibility: Evaluating the impact of a mandatory open data policy at the journal Cognition.

\item {\bf MacDonald, K.}, Marchman, V.A., Fernald, A., \& Frank, M.C. (under review). Adults and preschoolers seek visual information to support language comprehension in noisy environments.

\item Yoon, E.J.*, {\bf MacDonald, K.*}, Asaba M., Gweon, H., \& Frank, M.C. (under review). Balancing informational and social goals in active learning.

\item {\bf MacDonald, K.}, Blonder, A., Marchman, V.A., Fernald, A., \& Frank, M.C. (2017). An information-seeking account of eye movements during spoken and signed language processing. Proceedings of the 39th Annual Meeting of the Cognitive Science Society.

\item {\bf MacDonald, K.}, Yurovsky, D., \& Frank, M.C. (2017). Social cues modulate the representations underlying cross-situational learning. \emph{Cognitive Psychology}, 94, 67–84.

\item Frank, M.C., Lewis, M.L., \& {\bf MacDonald, K.}, (2016). A performance model for early word learning. Proceedings of the 38th Annual Meeting of the Cognitive Science Society.

\item {\bf MacDonald, K.}, \& Frank, M.C. (2016). When does passive learning improve the effectiveness of active learning? Proceedings of the 38th Annual Meeting of the Cognitive Science Society.

\item {\bf MacDonald, K.}, Yurovsky, D., \& Frank, M.C. (2015). Referential cues modulate attention and memory during cross-situational word learning. Proceedings of the 37th Annual Meeting of the Cognitive Science Society.

\item Barth, H., Bhandari, K., Garcia, J., {\bf MacDonald, K.}, \& Chase, E. (2014). Preschoolers trust novel members of accurate speakers? groups and judge them favorably. \emph{Quarterly Journal of Experimental Psychology}, 67, 872-883.

\item {\bf MacDonald, K.}, Schug, M., Chase, E. \& Barth, H. (2013). My People, Right or Wrong? Minimal Group Membership Disrupts Children's Selective Trust in Testimony. \emph{Cognitive Development} 28, 247-259.

\end{bibsection}

%%%%%%%%%%%%% TEACHING AND MENTORSHIP %%%%%%%%%%%%%%%

\section{Teaching and Mentorship}

\vspace{-.1in}

\begin{outerlist}

\item[] Teaching

\begin{innerlist}
\item[] Introduction to Statistical Methods (Head Teaching Assistant, 2017)\\
        Developmental Psychology (Instructor, 2016)\\
        Learning and Memory (Teaching Assistant, 2016)\\
        Developmental Psychology (Teaching Assistant, 2015)\\
        Introduction to Psychology (Teaching Assistant, 2014-2015)
\end{innerlist}

\item[] Undergraduate Research Assistants

\begin{innerlist}
\item[] Avivia Blonder - CSLI Research Intern\\
        Allison Dods - Symbolic Systems BS, honors thesis\\
        Melina Wailing - Psychology Research Assistant\\
        Tami Alade - HumBio Research Intern\\
        Hannah Slater - HumBio Research Intern
\end{innerlist}


\end{outerlist}

\vspace{.1in}


%%%%%%%%%%%%% PRESENTATIONS %%%%%%%%%%%%%%%

\section{Presentations}
\vspace{-.1in}

\begin{bibsection}

\item {\bf MacDonald, K.}, Marchman, V.A., Fernald, A., \& Frank, M.C. (2018). An information-seeking account of eye movements during spoken and signed language processing. Frisem presentation, Stanford University.

\item {\bf MacDonald, K.}, Blonder, A., Marchman, V.A., Fernald, A., \& Frank, M.C. (2017). An information-seeking account of eye movements during spoken and signed language processing. Oral presentation at the 39th Annual Meeting of the Cognitive Science Society, London, Eng.

\item {\bf MacDonald, K.}, Corina, D., Marchman, V., \& Fernald, A. (2016). Real-time language comprehension in American Sign Language. Workshop on Multimodal Multilingual Outcomes in Deaf and Hard-of-Hearing children, Stockholm, Sweden.

\item {\bf MacDonald, K.}, \& Frank, M.C. (2016). When does passive learning improve the effectiveness of active learning? Oral presentation at the 38th Annual Meeting of the Cognitive Science Society, Pasadena, CA.

\item {\bf MacDonald, K.}, Blonder, A., Marchman, V.A., Fernald, A., \& Frank, M.C. (2016). Speed-accuracy tradeoffs during real-time language comprehension in children learning English and American Sign Language. Poster presented at the biennial International Conference on Infant Studies, New Orleans, LA.

\item {\bf MacDonald, K.}, Yurovsky, D., \& Frank, M.C. (2015). Social cues modulate the strength of encoding alternative referents in cross-situational word learning. Conference talk submitted to the biennial meeting of the Society for Research in Child Development, Philadelphia, PA.

\item {\bf MacDonald, K.}, Corina, D., Marchman, V., \& Fernald, A. (2015). Age-related changes in children's real-time American Sign Language sentence processing. Conference talk submitted to the biennial meeting of the Society for Research in Child Development, Philadelphia, PA.

\item Bion, R., {\bf MacDonald, K.}, \& Fernald, A. (2013). M-o-o-s as cues: Young children map novel animal vocalizations to unfamiliar animals. Symposium submitted to the biennial meeting of the Society for Research in Child Development, Seattle, WA.

\item {\bf MacDonald, K.}, Corina, D., Marchman, V., \& Fernald, A. (2013). Real Time Processing of ASL in Deaf and Hearing Native-Signing Infants. Poster submitted to the biennial meeting of the Society for Research in Child Development, Seattle, WA.

\item {\bf MacDonald, K.}, Bion, R., Adams, K., Marchman, V., Hurtado, N., \& Fernald, A. (2012).   M-o-o-s as cues: Young children map novel animal vocalizations to unfamiliar animals. Poster presented at the biennial International Conference on Infant Studies, Minneapolis, MN.

\item {\bf MacDonald, K.}, Schug, M., \& Barth, H. (2011). My people, right or wrong? Minimal group membership disrupts children's selective trust in testimony. Poster presented at the biennial meeting of the Society for Research in Child Development, Montreal, Canada.

\item Barth, H., Garcia, J., Slusser, E., {\bf MacDonald, K.}, Acheampong, A., Kanjlia, S., \& Santiago, R. (2011). Proportional reasoning shapes children's number-line estimates. Conference abstract. Biennial meeting of the Society for Research in Child Development, Montreal, Canada.

\item Sullivan, J., {\bf MacDonald, K.}, Paladino, A., \& Barth, H. (2009). Children's mappings of number words to large numerosities. Conference abstract. Biennial meeting of the Society for Research in Child Development, Denver, CO.

\item Schug, M., Patalano, A., Barth, H., Shusterman, A., Herrig, E., \& {\bf MacDonald, K.} (2009). Group bias, statistical reasoning, and social judgments. Conference abstract. Biennial meeting of the Cognitive Development Society, San Antonio, TX.

\item {\bf MacDonald, K.} \& Barth, H. (2008). Learning the meaning of large number words. Poster presented at the Quantitative Analysis Center Poster Session, Wesleyan University, CT.

\end{bibsection}

%%%%%%%%%%%%%%%%%%%%%%%%%% Professional Service %%%%%%%%%%%%%%%%%%%%%%%%%%%%%

\section{Professional Service}

\begin{outerlist}
\item[] \emph{Student Representative}: Graduate Program Committee, Stanford University
\item[] \emph{Continuing professional memberships in}: Cognitive Science Society, International Society for Infant Studies, Society for Research in Child Development
\item[] \emph{Ad-hoc reviewer}: Journal of Experimental Child Psychology, Cognitive Development, Cognition, Developmental Psychology, Proceedings of the Cognitive Science Society, IEEE Transactions on Cognitive and Developmental Systems
\end{outerlist}

\halfblankline

\end{document}

%%%%%%%%%%%%%%%%%%%%%%%%%% End CV Document %%%%%%%%%%%%%%%%%%%%%%%%%%%%%

%----------------------------------------------------------------------%
% The following is copyright and licensing information for
% redistribution of this LaTeX source code; it also includes a liability
% statement. If this source code is not being redistributed to others,
% it may be omitted. It has no effect on the function of the above code.
%----------------------------------------------------------------------%
% Copyright (c) 2007, 2008, 2009, 2010, 2011 by Theodore P. Pavlic
%
% Unless otherwise expressly stated, this work is licensed under the
% Creative Commons Attribution-Noncommercial 3.0 United States License. To
% view a copy of this license, visit
% http://creativecommons.org/licenses/by-nc/3.0/us/ or send a letter to
% Creative Commons, 171 Second Street, Suite 300, San Francisco,
% California, 94105, USA.
%
% THE SOFTWARE IS PROVIDED "AS IS", WITHOUT WARRANTY OF ANY KIND, EXPRESS
% OR IMPLIED, INCLUDING BUT NOT LIMITED TO THE WARRANTIES OF
% MERCHANTABILITY, FITNESS FOR A PARTICULAR PURPOSE AND NONINFRINGEMENT.
% IN NO EVENT SHALL THE AUTHORS OR COPYRIGHT HOLDERS BE LIABLE FOR ANY
% CLAIM, DAMAGES OR OTHER LIABILITY, WHETHER IN AN ACTION OF CONTRACT,
% TORT OR OTHERWISE, ARISING FROM, OUT OF OR IN CONNECTION WITH THE
% SOFTWARE OR THE USE OR OTHER DEALINGS IN THE SOFTWARE.
%----------------------------------------------------------------------%
